\documentclass[12pt,onehalfspacing]{report}

\begin{document}

% JHU Dissertation title page
\thispagestyle{empty}

\begin{center}
{\large \MakeUppercase{\textbf{DESARROLLO DE UN SISTEMA DE LOCALIZACIÓN Y SEGURIDAD INFANTIL PARA PADRES DE FAMILIA}}}

\vspace{1in}

por \\
FELIX ANDRES ASELA GARCIA (460624)\\
VALENTINA PEÑUELA NAVAS (451607)\\
MICHAEL HERNANDEZ MORA (463810)

\vspace{1.5in}

El documento se ha elaborado siguiendo las pautas y directrices\\ establecidas para la investigación y desarrollo del proyecto.

\vspace{2in}

Bucaramanga, Santander\\       % location
MARZO 2024                  % date of submission goes here

\end{center}

\clearpage  % Nueva página para el contenido principal

\renewcommand{\contentsname}{Tabla de contenido}
\tableofcontents  % Agrega la tabla de contenido con el nuevo nombre

\clearpage  % Nueva página para el contenido principal

\chapter*{Introducción}
\addcontentsline{toc}{chapter}{Introducción}  % Añade manualmente la entrada en la tabla de contenido
La propuesta que se presenta a continuación tiene como objetivo principal desarrollar un sistema de localización y seguridad infantil para los padres, con la finalidad de brindarles la tranquilidad de conocer la ubicación de sus hijos en todo momento durante el trascurso de un trayecto, y recibir notificaciones al llegar al destino final.\\\\
Esta propuesta se enfoca en el diseño y desarrollo de un software especializado en la |localización en tiempo real de menores. Este sistema permitirá un seguimiento minucioso de los niños durante trayectos en los que no estén acompañados por sus padres, ofreciendo información detallada sobre su ubicación. El propósito principal de este sistema es aumentar la seguridad de los niños y facilitar la comunicación con los padres.\\\\
Para el desarrollo de este sistema se debe identificar 
la situación problema y la pregunta de investigación, ya que estos dos elementos son la base para la construcción del sistema planteado. Una correcta estructuración de la pregunta de investigación nos da a conocer con exactitud el problema que se quiere abordar y, por ende, poder resolverlo.\\\\
También se debe definir tanto el objetivo general como los objetivos específicos para el desarrollo del proyecto. Es fundamental destacar que es necesario mencionar las actividades que se deben realizar para cumplir de manera efectiva con cada uno de estos objetivos.\\\\
Al mismo tiempo se dan a conocer los antecedentes, donde se presenta el resultado de una búsqueda realizada para identificar y conocer proyectos que ya hayan abordado el mismo contexto del problema que hemos planteado. Finalmente, se expone la justificación del proyecto, que responde a preguntas fundamentales como: ¿Por qué se lleva a cabo? ¿Cuál es el propósito de este proyecto?.


\section*{Subsección 1}
Más texto para tu documento.

\section*{Subsección 2}
Otra subsección con más contenido.

\chapter*{Otra Sección}
Continuación del documento.

\end{document}
